%-------------------------------------------------------------------------------
%	SECTION TITLE
%-------------------------------------------------------------------------------
\cvsection{Опыт}


%-------------------------------------------------------------------------------
%	CONTENT
%-------------------------------------------------------------------------------
\begin{cventries}

  \cventry
    {Backend Developer} % Job title
    {Pyparse, Evotech} % Organization
    {Екатеринбург (Онлайн), Россия} % Location
    {Июль 2020 - Август 2021} % Date(s)
    {
      \begin{cvitems} % Description(s) of tasks/responsibilities
        \item {Проект для дистрибьюторов автозапчастей, основная задача - создать сервис для возможности поиска и покупки автозапчастей по списку сайтов онлайн магазинов, а также отслеживания заказов. }
        \item {Моей задачей была разработка воркера и парсера страниц на Python, которые отправяли запросы на сайты и доставали необходимую информацию. Задача усложнялась тем, что подавляющее большинство сайтов не имело открытых API. }
        \item {Изначально задача была решена другими людьми, но решение не имело полного функционала (только поиск запчастей) и работало слишком медленно (порядка 10-15 минут на один запрос), так как было написано с использованием Selenium. }
        \item {Использованные технологии: FastAPI, asyncio, BeautifulSoup, а также HTML, JavaScript для парсинга. }
        \item {В мои обязанности входило: разработка воркера, принимающего задачи от основного приложения, разработка парсеров для сайтов и логики для оформления заказов. }
        \item {В результате нам удалось создать расширяемую архитектуру проекта, при необходимости добавить новый источник заказов, разработчику не нужно брать во внимание функционал, выходящий за пределы парсинга. Удалось оптимизировать приложение, теперь на один запрос уходит около 30 секунд вместо 10-15 минут. }
      \end{cvitems}
    }

  \cventry
  {Backend Developer} % Job title
  {Software Project, Университет Иннополис} % Organization
  {Иннополис, Россия} % Location
  {Июнь - Июль 2022} % Date(s)
  {
    \begin{cvitems} % Description(s) of tasks/responsibilities
      \item {В рамках курса Software Project необходимо было разработать проект по теме "Reusable Learning App". }
      \item {Основной задачей проекта было предоставить шаблон приложения для образовательных материалов с целью дальнейшего расширения. В качестве примеров были выбраны приложения Duolingo, Brilliant. }
      \item {В мои задачи входила разработка API для мобильного приложения с возможностью регистрации и авторизации пользователей, добавления и модификации образовательных материалов, а также с системой курсов, системой опыта для отслеживания процесса изучения курсов. }
      \item {Использованные технологии: Django, Django REST Framework, PostgreSQL, Docker. }
    \end{cvitems}
  }

  \cventry
  {Backend Developer \& ML Engineer} % Job title
  {Capstone Project, Университет Иннополис \& Газпромстрой} % Organization
  {Иннополис, Россия} % Location
  {Июнь - Июль 2023} % Date(s)
  {
    \begin{cvitems} % Description(s) of tasks/responsibilities
      \item {В рамках курса Capstone Project необходимо было разработать сервис для компании Газпромстрой. }
      \item {Основная задача проекта заключалась в разработке системы для распознавания маркировки на трубах. }
      \item {Задача усложнялась тем, что маркировка была нанесена вручную разными людьми, разными инструментами (маркеры, стикеры и т. д.), помимо этого символы стирались в силу длительного пребывания в плохих условиях. }
      \item {В мои обязанности входило: подготовка данных для обучения модели распознавания, разметка данных, поиск наиболее подходящей модели, а также разработка API для функционала авторизации пользователей. }
      \item {Использованные технологии: OpenCV, CVAT, Pytorch, MMOCR, FastAPI, PostgreSQL, Docker. }
    \end{cvitems}
  }

  \cventry
    {Junior Backend Developer} % Job title
    {Инвиан} % Organization
    {Иннополис, Россия} % Location
    {Апрель 2024 - Июнь 2024} % Date(s)
    {
      \begin{cvitems} % Description(s) of tasks/responsibilities
        \item {Компания занимается анализом трафика на дорогах и управлением сигналов светофоров. }
        \item {Архитектура приложений соблюдала TDD. }
        \item {Использованные технологии: FastAPI, InfluxDB, telegraf, Dramatiq, Redis, pytest, pydantic }
        \item {Я занимался добавлением периодических задач в уже существующем сервисе с помощью telegraf и Dramatiq. }
        \item {Разрабывал сервис для создания матрицы корреспонденций для перекрестков в виде таблицы Excel, после чего он был обернут в API для возможности использования на фронтенде. Также добавил в API запросы для авторизации. }
        \item {Писал юнит-тесты для сервиса, взаимодействующего с InfluxDB. }
      \end{cvitems}
    }
  
  \cventry
  {Backend Developer} % Job title
  {TourManager} % Organization
  {Иннополис, Россия} % Location
  {Июнь - Сентябрь 2024} % Date(s)
  {
    \begin{cvitems} % Description(s) of tasks/responsibilities
      \item {Приложение для упрощения внутренней работы экскурсионных компаний. Позволяет менеджерам создавать экскурсии с полным набором необходимой информации, а экскурсоводы могут видеть свои экскурсии и их детали. }
      \item {В мои обязанности входило разработка нового бэкенда, так как старый имел ограниченный функционал и плохо расширяемую архитектуру.}
      \item {В результате мой была написана архитектура, соблюдающая Domain-Driven архитектуру и Чистую архитектуру Р. Мартина. }
      \item {Использованные технологии: FastAPI, adaptix, SQLAlchemy, PostgreSQL, pydantic, Redis. }
    \end{cvitems}
  }
\end{cventries}
