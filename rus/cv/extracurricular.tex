%-------------------------------------------------------------------------------
%	SECTION TITLE
%-------------------------------------------------------------------------------
\cvsection{Дополнительная активность}


%-------------------------------------------------------------------------------
%	CONTENT
%-------------------------------------------------------------------------------
\begin{cventries}

  \cventry
  {Graphics Engineer} % Affiliation/role
  {Геймджем Pyweek 35} % Organization/group
  {Онлайн} % Location
  {Март 2023} % Date(s)
  {
    \begin{cvitems} % Description(s) of experience/contributions/knowledge
      \item {Задача геймджема была создание игры на Python по теме "In the shadows". }
      \item {В мои обязанности входило разработка системы освещения и теней, в последствии взял задачу по механике преследования противниками главного игрока. }
      \item {Использованные технологии: OpenGL, Arcade. }
    \end{cvitems}
  }

%---------------------------------------------------------
\cventry
{Инженер машинного обучения} % Affiliation/role
{Хакатон "Цифровой прорыв. Сезон: Искусственный интеллект (ПФО)"} % Organization/group
{Пермь, Россия} % Location
{Октябрь 2023} % Date(s)
{
  \begin{cvitems} % Description(s) of experience/contributions/knowledge
    \item {Главной задачей был поиск решения задач обнаружения и отслеживания предметов твердых бытовых отходов на мультиспектральных снимках. }
    \item {Для решения проблемы обнаружния была использована модель YOLOv8. } 
    \item {Моей задачей было управлением команды внутри задач машинного обучения и предобработки данных. }
    \item {Исследовал актуальные модели и методы для решения задач обнаружения и отслеживания. }
    \item {Проводил анализ данных, так как они были поданы в нестандартном формате из-за каналов помимо RGB. }
    \item {Занимался аугментациями входных данных. }
    \item {Модифицировал модель YOLOv8 из библиотеки Ultralitycs для того, чтобы она могла принимать на вход одиннадцатиканальные изображения. }
    \item {После оценки экспертов наше решение заняло второе место. }
  \end{cvitems}
}
%---------------------------------------------------------
\cventry
{Backend инженер} % Affiliation/role
{Хакатон ВТБ MoreTECH 5.0} % Organization/group
{Онлайн} % Location
{Октябрь 2023} % Date(s)
{
  \begin{cvitems} % Description(s) of experience/contributions/knowledge
    \item {Отображение отделений банка ВТБ в удобном интерфейсе с целью облегчить поиск наиболее подходящего отделения. }
    \item {Наша команда использовала Flutter для написания мобильного приложения и FastAPI для написания бэкенда. } 
    \item {Для поиска оптимального отделения наша команда решила использовать две метрики: заполненность отделений и время, которое требуется для того, чтобы клиент мог добраться выбранным способом передвижения. }
    \item {Для измерения заполненности отделений мы предложили использовать изображения с камер наблюдения раз в двадцать минут, чтобы подсчитывать количество людей и вести статистику с случае отказа системы. }
    \item {В ходе работы я интегрировал микросервис, отслеживающий заполненность отделений банка. Для интеграции использовался RabbitMQ.}
    \item {Наша команда вышла в финал (топ-10), но не заняла призовых мест. }
  \end{cvitems}
}
%---------------------------------------------------------
\cventry
{Инженер машинного обучения} % Affiliation/role
{Хакатон "Цифровой прорыв. Сезон: Искусственный интеллект (Международный)"} % Organization/group
{Онлайн} % Location
{Ноябрь 2023} % Date(s)
{
  \begin{cvitems} % Description(s) of experience/contributions/knowledge
    \item {Задачей хакатона было определение на изображении объектов инфраструктуры на снимках со спутника. }
    \item {Я управлял командой внутри задач машинного обучения и предобработки данных. }
    \item {В ходе хакатона я изучал литературу, чтобы протестировать нестандартные модели для нашей задачи, к сожалению, для методов, предложенных в исследованиях, нам не хватило времени и вычислительных ресурсов, поэтому было решено использовать дообучение готовых моделей. }
    \item {Нашим решением была обученная модель Unet. } 
  \end{cvitems}
}
%---------------------------------------------------------
\cventry
{Инженер машинного обучения} % Affiliation/role
{Хакатон IT Inno Hack} % Organization/group
{Онлайн} % Location
{Сентябрь 2024} % Date(s)
{
  \begin{cvitems} % Description(s) of experience/contributions/knowledge
    \item {Перед нами была поставлена задача объединения записей. }
    \item {Данными являлись три базы данных с информацией о людях, было необходимо объединить записи, относящиеся к одному и тому же человеку. }
    \item {Данные были повреждены: в некоторых записях отсутствовали поля, в других были опечатки, а иногда слова могли дублироваться. }
    \item {Немаловажной проблемой являлось то, что было необходимо проанализировать 11 миллионов записей за 20 минут. }
    \item {Я занималися исследованием литературы, относящейся к нашей теме и методами, которыми пользуются для решения задачи Record Linkage. }
    \item {Решением проблемы было использование EM-алгоритма с расстоянием Левенштейна.}
    \item {После оценки решений наша команда заняла четвертое место с минимальным отрывом от призеров. }
  \end{cvitems}
}
%---------------------------------------------------------
\cventry
{Инженер машинного обучения} % Affiliation/role
{Хакатон "Цифровой прорыв. Сезон: Искусственный интеллект (ПФО)"} % Organization/group
{Нижний Новгород, Россия} % Location
{Октябрь 2024} % Date(s)
{
  \begin{cvitems} % Description(s) of experience/contributions/knowledge
    \item {Задача заключалась в обнаружении затопленных регионов и объектов инфраструктуры на мультиспектральных изображениях со спутников. }
    \item {Помимо этого одной из метрик был подсчет количества домов, попавших в зону затопления. }
    \item {Я занимался изучением исследований, относящихся к нашей задаче. }
    \item {В ходе решения, я реализовал алгоритм, использующий особенности отражения поверхности через модифицированную матрицу коокурентности для цветных изображений, а также алгоритм для обучения генеративно-состязательной сети}
    \item {В качестве решения мы предоставили генеративно-состязательную сеть с моделью Unet в качестве генератора. } 
    \item {По итогам наша команда заняла седьмое место. }
  \end{cvitems}
}
\end{cventries}
