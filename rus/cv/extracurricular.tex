%-------------------------------------------------------------------------------
%	SECTION TITLE
%-------------------------------------------------------------------------------
\cvsection{Внеурочные занятия}


%-------------------------------------------------------------------------------
%	CONTENT
%-------------------------------------------------------------------------------
\begin{cventries}

%---------------------------------------------------------
\cventry
{Инженер машинного обучения} % Affiliation/role
{Хакатон "Цифровой прорыв. Сезон: Искусственный интеллект (ПФО)"} % Organization/group
{Пермь, Россия} % Location
{Октябрь 2023} % Date(s)
{
  \begin{cvitems} % Description(s) of experience/contributions/knowledge
    \item {Главной задачей был поиск решения задач обнаружения и отслеживания предметов твердых бытовых отходов на мультиспектральных снимках. }
    \item {Для решения проблемы обнаружния была использована модифицированная модель YOLOv8. } 
    \item {Для решении задачи отслеживания был использован алгоритм DeepSort. }
    \item {Моей задачей было управлением команды внутри задач машинного обучения и предобработки данных. }
  \end{cvitems}
}
%---------------------------------------------------------
\cventry
{Backend инженер} % Affiliation/role
{Хакатон ВТБ MoreTECH 5.0} % Organization/group
{Онлайн} % Location
{Октябрь 2023} % Date(s)
{
  \begin{cvitems} % Description(s) of experience/contributions/knowledge
    \item {Отображение отделений банка ВТБ в удобном интерфейсе с целью облегчить поиск наиболее подходящего отделения. }
    \item {Наша команда использовала Flutter для написания мобильного приложения и FastAPI для написания бэкенда. } 
    \item {В ходе работы я интегрировал микросервис, отслеживающий заполненность отделений банка. Для интеграции использовался RabbitMQ. }
  \end{cvitems}
}
%---------------------------------------------------------
\cventry
{Инженер машинного обучения} % Affiliation/role
{Хакатон "Цифровой прорыв. Сезон: Искусственный интеллект (Международный)"} % Organization/group
{Онлайн} % Location
{Ноябрь 2023} % Date(s)
{
  \begin{cvitems} % Description(s) of experience/contributions/knowledge
    \item {Задачей хакатона было определение на изображении объектов инфраструктуры на снимках со спутника. }
    \item {Я управлял командой внутри задач машинного обучения и предобработки данных. }
    \item {В ходе хакатона я изучал литературу, чтобы протестировать нестандартные модели для нашей задачи. }
    \item {Нашим решением была обученная модель Unet. } 
  \end{cvitems}
}
%---------------------------------------------------------
\cventry
{Инженер машинного обучения} % Affiliation/role
{Хакатон IT Inno Hack} % Organization/group
{Онлайн} % Location
{Сентябрь 2024} % Date(s)
{
  \begin{cvitems} % Description(s) of experience/contributions/knowledge
    \item {Перед нами была поставлена задача объединения записей. }
    \item {Данными являлись три базы данных с поврежденной информацией о людях, было необходимо объединить записи, относящиеся к одному и тому же человеку. }
    \item {Основной проблемой являлось то, что было необходимо проанализировать 11 миллионов записей за 20 минут. }
    \item {Я занималися исследованием литературы, относящейся к нашей теме. }
    \item {Решением проблемы было использование EM-алгоритма с расстоянием Левенштейна.}
  \end{cvitems}
}
%---------------------------------------------------------
\cventry
{Инженер машинного обучения} % Affiliation/role
{Хакатон "Цифровой прорыв. Сезон: Искусственный интеллект (ПФО)"} % Organization/group
{Нижний Новгород, Россия} % Location
{Октябрь 2024} % Date(s)
{
  \begin{cvitems} % Description(s) of experience/contributions/knowledge
    \item {Задача заключалась в обнаружении затопленных регионов и объектов инфраструктуры на мультиспектральных изображениях со спутников. }
    \item {Я занимался изучением исследований, относящихся к нашей задаче. }
    \item {В качестве решения мы обучили генеративно-состязательную сеть с моделью Unet в качестве генератора. } 
  \end{cvitems}
}
\end{cventries}
